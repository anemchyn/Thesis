%%=============================================================================
%% Samenvatting
%%=============================================================================

% TODO: De "abstract" of samenvatting is een kernachtige (~ 1 blz. voor een
% thesis) synthese van het document.
%
% Een goede abstract biedt een kernachtig antwoord op volgende vragen:
%
% 1. Waarover gaat de bachelorproef?
% 2. Waarom heb je er over geschreven?
% 3. Hoe heb je het onderzoek uitgevoerd?
% 4. Wat waren de resultaten? Wat blijkt uit je onderzoek?
% 5. Wat betekenen je resultaten? Wat is de relevantie voor het werkveld?
%
% Daarom bestaat een abstract uit volgende componenten:
%
% - inleiding + kaderen thema
% - probleemstelling
% - (centrale) onderzoeksvraag
% - onderzoeksdoelstelling
% - methodologie
% - resultaten (beperk tot de belangrijkste, relevant voor de onderzoeksvraag)
% - conclusies, aanbevelingen, beperkingen
%
% LET OP! Een samenvatting is GEEN voorwoord!

%%---------- Nederlandse samenvatting -----------------------------------------
%
% TODO: Als je je bachelorproef in het Engels schrijft, moet je eerst een
% Nederlandse samenvatting invoegen. Haal daarvoor onderstaande code uit
% commentaar.
% Wie zijn bachelorproef in het Nederlands schrijft, kan dit negeren, de inhoud
% wordt niet in het document ingevoegd.

\IfLanguageName{english}{%
\selectlanguage{dutch}
\chapter*{Samenvatting}
\selectlanguage{english}
}{}

%%---------- Samenvatting -----------------------------------------------------
% De samenvatting in de hoofdtaal van het document

\chapter*{\IfLanguageName{dutch}{Samenvatting}{Abstract}}

De stijgende voedselprijzen in België leggen een groeiende financiële druk op studenten met een beperkt budget en beperkte mobiliteit. Bestaande prijsvergelijkingstools houden onvoldoende rekening met praktische beperkingen van winkelen, zoals reisafstand en het aantal winkels dat een consument bereid is te bezoeken. Bovendien baseren veel tools zich op barcodevergelijking, waardoor huismerken vaak buiten de scope worden gelaten. Daarnaast staan veel van deze diensten of hun extra functies achter een betaalmuur.

Dit onderzoek richt zich op het ontwerpen van een systeem dat consumenten, met name studenten in Gent, ondersteunt om weloverwogen beslissingen te kunnen nemen over hun boodschappen. De centrale onderzoeksvraag luidt: “Hoe kan een transparant en schaalbaar systeem worden ontworpen en ontwikkeld om automatisch supermarktprijzen in België te verzamelen, matchen en vergelijken, rekening houdend met beperkingen, zoals afstand en het aantal te bezoeken winkels?”

Het voorgestelde prototype wordt ontwikkeld in Python en Django, waarbij gebruik wordt gemaakt van webscrapingstechnieken om prijsgegevens van Belgische supermarktwebsites te verzamelen. De gegevens worden opgeslagen in een PostgreSQL-database. Gebruikers kunnen een boodschappenlijst invoeren en een maximale reisafstand of een limiet voor het aantal winkels instellen. Vervolgens wordt het meest kostefficiënte aankoopplan door het systeem berekend. De gegenereerde voorstellen worden geëvalueerd door hun besparingen te vergelijken met een aankoop van alle boodschappen in één winkel, op basis van vooraf gedefinieerde boodschappenlijsten. De resultaten zijn merkbare besparingen over alle voorgestelde boodschappenlijsten. 

Dit project draagt ​​bij aan een realistisch en toegankelijk prijsvergelijkingssysteem gericht op studenten door technische efficiëntie te combineren met consumentgerichte beperkingen. Dit stelt studenten in staat bewuster en voordeliger te winkelen.