%%=============================================================================
%% Samenvatting
%%=============================================================================

% TODO: De "abstract" of samenvatting is een kernachtige (~ 1 blz. voor een
% thesis) synthese van het document.
%
% Een goede abstract biedt een kernachtig antwoord op volgende vragen:
%
% 1. Waarover gaat de bachelorproef?
% 2. Waarom heb je er over geschreven?
% 3. Hoe heb je het onderzoek uitgevoerd?
% 4. Wat waren de resultaten? Wat blijkt uit je onderzoek?
% 5. Wat betekenen je resultaten? Wat is de relevantie voor het werkveld?
%
% Daarom bestaat een abstract uit volgende componenten:
%
% - inleiding + kaderen thema
% - probleemstelling
% - (centrale) onderzoeksvraag
% - onderzoeksdoelstelling
% - methodologie
% - resultaten (beperk tot de belangrijkste, relevant voor de onderzoeksvraag)
% - conclusies, aanbevelingen, beperkingen
%
% LET OP! Een samenvatting is GEEN voorwoord!

%%---------- Nederlandse samenvatting -----------------------------------------
%
% TODO: Als je je bachelorproef in het Engels schrijft, moet je eerst een
% Nederlandse samenvatting invoegen. Haal daarvoor onderstaande code uit
% commentaar.
% Wie zijn bachelorproef in het Nederlands schrijft, kan dit negeren, de inhoud
% wordt niet in het document ingevoegd.

\IfLanguageName{english}{%
\selectlanguage{dutch}
\chapter*{Samenvatting}
\selectlanguage{english}
}{}

%%---------- Samenvatting -----------------------------------------------------
% De samenvatting in de hoofdtaal van het document

\chapter*{\IfLanguageName{dutch}{Samenvatting}{Abstract}}

De stijgende voedselprijzen in België leggen een toenemende druk op studenten met een beperkt budget en beperkte mobiliteit.
Bestaande prijsvergelijkingstools zijn niet geschikt voor deze doelgroep, omdat ze geen rekening houden met reisafstanden of
het aantal winkels dat een consument bereid is te bezoeken. Wat nog een probleem vormt is dat deze tools gebruiken barcodes om prijzen te vergelijken, wat betekent dat A-merken worden genegeerd of nauwelijks gezien. Bovendien zijn sommige daarvan betaald. 
Dit onderzoek is gericht op het ontwerpen van een systeem dat klanten helpt weloverwogen beslissingen te nemen over hun boodschappen met ee duidelijke inzicht op prijzen, met de focus op studenten in Gent. De belangrijkste onderzoeksvraag is: Hoe kan een transparant en schaalbaar systeem worden ontworpen en ontwikkeld om automatisch supermarktprijzen in België te verzamelen, te matchen en te vergelijken, en tegelijkertijd
rekening te houden met de beperkingen van de afstand en het aantal winkels van klanten?

Het prototype zal worden geïmplementeerd in Python en Django, met behulp van webscraping bibliotheken om prijsgegevens te verzamelen van supermarktwebsites en een PostgreSQL-database om de gegevens te bewaren. Gebruikers kunnen een boodschappenlijstje invoeren en een
maximale reisafstand of aantal winkels opgeven, waarna het systeem het meest kosteneffectieve boodschappenplan berekent.
De evaluatie vergelijkt de totale kosten van vooraf gedefinieerde winkelwagentjes voor studenten tussen één winkel en geoptimaliseerde opties voor meerdere winkels. Verwachte resultaten zijn onder meer meetbare kostenbesparingen.

Dit project draagt ​​bij aan realistischere en toegankelijkere prijsvergelijkingssystemen door technische efficiëntie te integreren met consumentgerichte beperkingen, waardoor studenten weloverwogen en betaalbare winkelkeuzes kunnen maken.
