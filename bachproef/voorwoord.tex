%%=============================================================================
%% Voorwoord
%%=============================================================================

\chapter{Woord vooraf}%
\label{ch:voorwoord}

%% TODO:
%% Het voorwoord is het enige deel van de bachelorproef waar je vanuit je
%% eigen standpunt (``ik-vorm'') mag schrijven. Je kan hier bv. motiveren
%% waarom jij het onderwerp wil bespreken.
%% Vergeet ook niet te bedanken wie je geholpen/gesteund/... heeft

Einde van de weg. Het laatste grote project achter de rug. En wat voor een project was het ook om mee te mogen afstuderen. Naast de vele frustraties heeft het mij ook ontzettend veel geleerd over scraping, asynchroon programmeren en prijzen in de winkel.

Ik wil hier mijn co-promotor Jens Pots bedanken voor zijn leiding en steun wanneer geen idee had wat een verdere stap moet zijn. Ik wil mijn promotor Leen Vuyge bedanken voor haar geduld, positieve geest en constructieve feedback. In het bijzonder wil ik Jozef, mijn verloofde, bedanken voor het vele leeswerk en suggesties, alsook vele tassen koffie.

\emph{De auteur is geen moedertaalspreker van het Nederlands. Voor taalkundige correcties en herformuleringen van de tekst werd gebruikgemaakt van een AI-gebaseerde taalassistent. De inhoud, structuur en conclusies van dit werk zijn volledig van de auteur zelf.}